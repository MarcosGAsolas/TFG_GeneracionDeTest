\capitulo{1}{Introducción}

Descripción del contenido del trabajo y del estructura de la memoria y del resto de materiales entregados.

Lo que se pretende realizar en este trabajo, es una página web para la generacion de test unitarios de un código que el usuario haya solicitado.

Este trabajo se va a dividir en dos bloques grandes, el primero de ellos es la parte de front, es decir, la generacion de la página web, y la parte de back, es decir, toda la parte de la funcionalidad de la página web.

\section*{Entregas}

Las entregas se van a realizar todos los domingos de las semanas, es decir, se van a realizar springs de una semana, para que el trabajo que se haya ido realizando durante la semana se pueda visualizar a lo largo de la semana siguiente.

\section*{Front}

Es la parte que se va a realizar en primera instancia, ya que es la parte más visual del proyecto y la que el usuario puede ir viendo como va avanzando el proyecto. 

El objetivo de realizar primero la pagina web, es dar al usuario un feedback de lo que se esta realizando he ir añadiendo funcionalidades poco a poco, pero que el usaurio pueda ir probando a la vez lo que se esta haciendo, ya que, si primero hacemos la parte funcional del proyecto, el usuario no va a saber que se esta realizando y con que frecuencia, sin embargo, de esta manera, el usuario, como bien hemos dicho anteriormente, puede ir probando las funcionalidades que se van añadiendo de manera simultanea.

\section*{Back}

Es la ultima parte que se va a realizar, para ir añadiendo funcionalidades a la pagina web que ya tengamos hecha.

El objetivo de la parte de back es implementar la funcionalidad de la generacion de los test y del porcentaje que estos han cubierto del código pasado, ademas de visualizar el código del cual no se ha realizado test para que el usuario pueda, o bien volver a generar los test, o bien crearlos el mismo.
