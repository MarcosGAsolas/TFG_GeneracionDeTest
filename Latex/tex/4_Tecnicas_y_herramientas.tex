\capitulo{4}{Técnicas y herramientas}

Esta parte de la memoria tiene como objetivo presentar las técnicas metodológicas y las herramientas de desarrollo que se han utilizado para llevar a cabo el proyecto. Si se han estudiado diferentes alternativas de metodologías, herramientas, bibliotecas se puede hacer un resumen de los aspectos más destacados de cada alternativa, incluyendo comparativas entre las distintas opciones y una justificación de las elecciones realizadas. 
No se pretende que este apartado se convierta en un capítulo de un libro dedicado a cada una de las alternativas, sino comentar los aspectos más destacados de cada opción, con un repaso somero a los fundamentos esenciales y referencias bibliográficas para que el lector pueda ampliar su conocimiento sobre el tema.

\section*{Herramientas utilizadas}

Las herramientas que se van a utilizar en el tfg van a ser las siguientes:
    \begin{itemize}
        \item \textbf{Dbeaver:} para la base de datos, he elegido este entorno para manejar base de datos debido a que ofrece una gran variedad de bases de datos ademas de que tiene un manejo muy sencillo y muy intuitivo a la hora de crear la base de datos y gestionarla, pudiendo crear y eliminar sin necesidad de implementar código, ademas, de que si tienes varios entornos de bases de datos, con dbeaver puede generar diferentes scripts de estos sin ninguna dificultad y sin tener que cambiar de entorno de trabajo
        \item \textbf{Visual Studio 2022:} para generar la parte del back con el framework de asp.net, he elegido esta technologia debido a que, aunque no sea la más utilizada, para mi, el lenguaje y el entorno en el que se trabaja es mucho mas intuitivo y da menos problemas a la hora de inicializar el entorno de trabajo con respecto a otros lenguajes y otras technologias, ademas de que es una de las más utilizadas junto con spring
        \item \textbf{Latex:} para generar el documento del tfg, he elegido esta herramienta ya que, si la comparamos con word o con otro tipo de programas para crear un documento, no ofrecen las mismas posibilidades y ademas, es la herramienta por excelencia para generar documentos de artículos y de trabajos
        \item \textbf{Herramienta de Ia:} para crer la ia para generar los test unitarios (todavia no la se)
        \item \textbf{Angular:} para generar la pagina web, me parece una herramienta muy adecuada para realizar cualquier proyecto web, si bien es cierto que puede ser que la curva de aprendizaje puede ser más pronunciada que las demas, es una de las mejores opciones para crear un proyecto, y que ademas, cuenta con un sistema cli con el cual puedes crear diferentes componentes autogenerados sin la necesidad de que tengas que estar programándo y viendo que tienes que modificar y asi te evitas tener fallos 
        \item \textbf{Visual Studio Code 2022:} para genera la pagina web y generar el documento latex, me he decidido a trabajar con visual studio code porque desde esta herramienta puedo trabajar simultaneamente con latex y con angular, ademas de que puede ofrecer diferentes herramientas para facilitar el trabajo en este entorno {Agregar herramientas que se hayan utilizado}
    \end{itemize}




